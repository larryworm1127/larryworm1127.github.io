%%%%%%%%%%%%%%%%%
% This is an example CV created using altacv.cls (v1.1.4, 27 July 2018) written by
% LianTze Lim (liantze@gmail.com), based on the
% Cv created by BusinessInsider at http://www.businessinsider.my/a-sample-resume-for-marissa-mayer-2016-7/?r=US&IR=T
%
%% It may be distributed and/or modified under the
%% conditions of the LaTeX Project Public License, either version 1.3
%% of this license or (at your option) any later version.
%% The latest version of this license is in
%%    http://www.latex-project.org/lppl.txt
%% and version 1.3 or later is part of all distributions of LaTeX
%% version 2003/12/01 or later.
%%%%%%%%%%%%%%%%

%% If you want to use \orcid or the
%% academicons icons, add "academicons"
%% to the \documentclass options.
%% Then compile with XeLaTeX or LuaLaTeX.
% \documentclass[10pt,a4paper,academicons]{altacv}

%% Use the "normalphoto" option if you want a normal photo instead of cropped to a circle
% \documentclass[10pt,a4paper,normalphoto]{altacv}

\documentclass[10pt,a4paper]{altacv}

%% AltaCV uses the fontawesome and academicon fonts
%% and packages.
%% See texdoc.net/pkg/fontawecome and http://texdoc.net/pkg/academicons for full list of symbols.
%% When using the "academicons" option,
%% Compile with LuaLaTeX for best results. If you
%% want to use XeLaTeX, you may need to install
%% Academicons.ttf in your operating system's font %% folder.


% Change the page layout if you need to
\geometry{left=1cm,right=9cm,marginparwidth=6.8cm,marginparsep=1.2cm,top=1cm,bottom=1cm}

% Change the font if you want to.

% If using pdflatex:
\usepackage[utf8]{inputenc}
\usepackage[T1]{fontenc}
\usepackage[default]{lato}
\usepackage{hyperref}
\usepackage{fontawesome}

% If using xelatex or lualatex:
% \setmainfont{Lato}

% Change the colours if you want to
\definecolor{VividPurple}{HTML}{3E0097}
\definecolor{SlateGrey}{HTML}{2E2E2E}
\definecolor{LightGrey}{HTML}{666666}
\colorlet{heading}{VividPurple}
\colorlet{accent}{VividPurple}
\colorlet{emphasis}{SlateGrey}
\colorlet{body}{black}

% Change the bullets for itemize and rating marker
% for \cvskill if you want to
\renewcommand{\itemmarker}{{\small\textbullet}}
\renewcommand{\ratingmarker}{\faCircle}

\begin{document}
\name{Larry Shi}
\tagline{Software Developer}
\personalinfo{%
  % Not all of these are required!
  % You can add your own with \printinfo{symbol}{detail}
  \email{\href{mailto:shilerong@gmail.com}{shilerong@gmail.com}}
  \phone{343-777-8457}
  \homepage{\href{https://larryworm1127.github.io/}{larryworm1127.github.io}}
  \linkedin{\href{https://www.linkedin.com/in/larry-shi-11479914b/}{linkedin.com/in/larry-shi-11479914b/}}
  \github{\href{https://github.com/larryworm1127}{github.com/larryworm1127}}
}

%% Make the header extend all the way to the right, if you want.
\begin{fullwidth}
\makecvheader
\end{fullwidth}

%% Depending on your tastes, you may want to make fonts of itemize environments slightly smaller
\AtBeginEnvironment{itemize}{\small}

%% Include the sidebar
\marginpar{\vspace*{-85pt}\raggedright\cvsection[]{Skills}
\begin{itemize}
\item Strong experience in Python web frameworks (Django, Flask), and some experience with NumPy, Hypothesis, Pandas, and Matplotlib.
\item Knowledgeable in common design patterns and development practices, such as SOLID, MVP, and REST API’s
\item Professional experience with Git, Scrum, automated testing, Unix, with some knowledge of networking.
\item Experienced in Android development with both Java and Kotlin ecosystem.
\item Some experience with CI/CD with CircleCI, and deployment with Heroku.
\item Some experience with C, PostgreSQL, SQLite, Latex, React, Angular
\end{itemize}


\cvsection{Education}

\cvevent{\faGraduationCap \, B.Sc., Computer Science}{University of Toronto}{2018 -- 2022}{Toronto, ON}


Current CS/CS Related Courses \\[4pt]
\begin{itemize}
\item \textbf{Software Design (with Java)}
\item \textbf{Computer Organization}
\item \textbf{Introduction to Theory of Computation}
\item \textbf{Software Tools and Systems Programming (with Bash and C)}
\item \textbf{Introduction to Databases (with PostgreSQL)}
\item \textbf{Data Structure and Analysis}
\item \textbf{Probability with Computer Applications (with R)}
\end{itemize}


\cvsection{Awards \& Certificates}
\cvevent{"Learn to Program: The Fundamentals" Statement of Accomplishment}{University of Toronto MOOC}{January 2014}{Ottawa, ON}

\divider

\cvevent{"An Introduction to Interactive Programming in Python" Statement of Accomplishment With Distinction}{Rice University MOOC}{June 2014}{Ottawa, ON}

\divider

\cvevent{Silver Medal (>90\% average)}{Earl of March S.S.}{2015 -- 2018}{Ottawa, ON}
}

%% Provide the file name containing the sidebar contents as an optional parameter to \cvsection.
%% You can always just use \marginpar{...} if you do
%% not need to align the top of the contents to any
%% \cvsection title in the "main" bar.
\cvsection[]{Experience}

\cvevent{Software Developer in Test - Intern}{Merchant Link}{May 2019 -- August 2019}{Ottawa, ON}
\begin{itemize}
\item Used Python to create scripts that simulates behaviors of a pinpad to help automate testing of company product.
\item Performed stress testing on company product to evaluate its ability to process transactions.
\item Used Python to create an experimental tool that captures product network activities and diagnose for pinpad issues.
\end{itemize}


\cvsection[]{Personal Projects}

\cvevent{\href{https://github.com/larryworm1127/PyBoardGame}{PyBoardGame}}{}{July 2018 -- September 2018}{Ottawa, ON}
\begin{itemize}
    \item A web project written in Python (Flask) that allows users to play simple board games on the website. Including Tic Tac Toe and Sudoku.
    \item A computer player is implemented using minimax algorithm in Python while the rest of game logic is implemented in JavaScript.
    \item The website also contains well documented game strategies pages.
\end{itemize}

\divider

\cvevent{\href{https://github.com/larryworm1127/nba\_daily}{NBA Daily}}{}{Apirl 2019 -- Ongoing}{Ottawa, ON}
\begin{itemize}
    \item A web project written in Python (Django) that scraps NBA stats from NBA website using API and displays the data in a more accessible way.
    \item Used SQLite alongside Django model to store and access all scapped data.
    \item Currently trying to integrate React with Django Rest Framework to create better looking and functinal web pages.
\end{itemize}

\divider

\cvevent{\href{https://github.com/larryworm1127/AdventureOfPost}{Android Game}}{}{October 2019 -- December 2019}{Toronto, ON}
\begin{itemize}
	\item An Android app that contains 3 distinctive games, a functional user authentication system, as well as game save/resume feature.
  \item Learned Android development in a week and built the app using Java, Gradle, and SQLite.
  \item Employed SOLID principles, MVP architecture and other object-oriented design patterns.
	\item Designed the project, oral presentation, UML diagrams in collaboration with five other peers.
\end{itemize}


\cvsection[]{Extracurricular Activities}

\cvevent{Volunteer}{Stillwater Creek Retirement Home}{July 2014 -- June 2018}{Ottawa, ON}
\begin{itemize}
    \item Perform music with a group named "Music in Motion" in the retirement home.
    \item Organized "Music in Motion" performances.
    \item Help kitchen serve out ice creams to seniors in the retirement home.
\end{itemize}

\end{document}
